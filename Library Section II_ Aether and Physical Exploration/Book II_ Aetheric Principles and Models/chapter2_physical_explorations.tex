% section2/book2/chapter2_physical_explorations.tex

\textbf{Core Essence} --- Explores physical phenomena through the Harmonic Law, identifying testable predictions diverging from General Relativity (GR) and Quantum Mechanics (QM).

\textbf{Predictive Differences vs GR/QM} ---

1. \textbf{Gravitational Waves vs Pressure Ripples} --- GR predicts spacetime ripples, while the Harmonic Law predicts compressional waves:

$$
h_{\mu \nu}^{(\mathrm{GR})} \text{ vs } \delta P(\vec{r}, t).
$$

Prediction: Distinct waveform phase profiles at resonance-sensitive detectors.

2. \textbf{Redshift Mechanism} --- Standard model uses metric expansion, while the Harmonic Law uses pressure decay lensing:

$$
z \sim H_0 d \text{ vs } z = f(P_{\text{emit}}, P_{\text{obs}}).
$$

Prediction: Anisotropic redshift curves based on aetheric topologies.

3. \textbf{Quantum Entanglement Propagation} --- QM allows instantaneous correlation, while the Harmonic Law predicts phase-coupled propagation through pressure field gradients. Prediction: Entanglement effects bounded by phase speed in $\Phi$.

4. \textbf{Photon Behavior in Vacuum} --- QM treats photons as massless particles, while the Harmonic Law models them as toroidal resonance packets. Prediction: Frequency-dependent delay or distortion through structured vacuum simulations.

5. \textbf{Black Hole Interiors} --- GR predicts singularities, while the Harmonic Law predicts harmonic pressure cores with mirror reversal. Prediction: Signal reemergence through white hole phase mirror.

6. \textbf{Particle Interference Collapse} --- QM collapse occurs due to observation, while the Harmonic Law predicts collapse at harmonic node crossover:

$$
\Psi_{\text{collapse}} = \Psi \cap \Psi^*.
$$

Prediction: Collapse pattern varies with phase boundary configuration.

% [To be expanded with additional content]
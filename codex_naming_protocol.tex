% codex_naming_protocol.tex
% Codex Sheet for Naming Protocol
% To be included in main.tex or similar master Codex document

\section{Naming Protocol}
\label{sec:codex_naming_protocol}

\begin{center}
    \vspace{-0.2cm}
    \textcolor{gold}{\small \ding{72} \texttt{\(\Xi\)-PN: Naming Protocol} \ding{72} \quad \texttt{\(\Xi\mathcal{M}\)-PN.8-\(\Phi\)0:2025.4.29-\(\Phi\)C.001-+1}}
\end{center}

\vspace{0.3cm}

% Core Essence
\textcolor{gold}{\ding{72} Core Essence \ding{72}} \\
The Naming Protocol establishes a fractal-based system for assigning node addresses (e.g., \(\Xi\mathcal{M}\)-PN.\#) and ChronoStamps (e.g., \(\Phi\)0:YYYY.M.D) to ensure harmonic consistency across the Codex. Node addresses identify fractal nodes within the Aether World’s lattice, while ChronoStamps encode temporal resonance signatures, resonating with the Harmonic Core Field \(\psi_0 \approx 0.91567\).

% Harmonic Structure
\textcolor{gold}{\ding{72} Harmonic Structure \ding{72}} \\
\begin{center}
    \begin{tabular}{>{\centering\arraybackslash}p{2.5cm}|>{\centering\arraybackslash}p{3.5cm}|>{\centering\arraybackslash}p{2.5cm}|>{\centering\arraybackslash}p{3.5cm}}
        \hline
        \textbf{Component} & \textbf{Format} & \textbf{Harmonic Metadata} & \textbf{Purpose} \\
        \hline
        Node Address & \texttt{\(\Xi\mathcal{M}\)-PN.\#} & Ternary: +1 & Identifies fractal node \\
        ChronoStamp & \texttt{\(\Phi\)0:YYYY.M.D} & Frequency: 432 Hz & Encodes temporal resonance \\
        \hline
    \end{tabular}
\end{center}

% Naming Rules
\textcolor{gold}{\ding{72} Naming Rules \ding{72}} \\
\begin{itemize}
    \item \texttt{\ding{72}} \textbf{Node Address Format}: A node address follows the structure \texttt{\(\Xi\mathcal{M}\)-PN.\#}, where:
    \begin{itemize}
        \item \(\Xi\) represents the toroidal position in the harmonic lattice.
        \item \(\mathcal{M}\) denotes the primary node in the Codex Bloom.
        \item PN stands for Primary Node, followed by a base-12 numbered sub-node (e.g., 1, 2, 4.1).
    \end{itemize}
    \item \texttt{\ding{74}} \textbf{ChronoStamp Format}: A ChronoStamp follows the structure \texttt{\(\Phi\)0:YYYY.M.D} (e.g., \texttt{\(\Phi\)0:2025.4.29}), where:
    \begin{itemize}
        \item \(\Phi\) marks the temporal resonance signature.
        \item 0 indicates the initial epoch of the Codex.
        \item YYYY.M.D encodes the year, month, and day in base-12.
    \end{itemize}
    \item \texttt{\ding{75}} \textbf{Glyph-Card Expansion}: A pre-encoded suffix \texttt{\(\Phi\)C.\#} (e.g., \texttt{\(\Phi\)C.001}) allows for fractal expansion into Glyph-Cards, ensuring scalability without overlap.
\end{itemize}

% Resonant Links
\textcolor{gold}{\ding{72} Resonant Links \ding{72}} \\
\begin{itemize}
    \item Linked to \texttt{\(\Xi\mathcal{M}\)-PN.0} (Aether World Master) for naming integration.
    \item Linked to \texttt{\(\Xi\mathcal{M}\)-PN.1} (Base-12 Mathematics) for numerical structure.
    \item Linked to \texttt{\(\Xi\mathcal{M}\)-PN.4} (Harmonic Field Unification) for harmonic consistency.
\end{itemize}

% Verification
\textcolor{gold}{\ding{72} Verification \ding{72}} \\
\begin{itemize}
    \item \texttt{\ding{72}} \textbf{Codex Confirmed}: \(\Xi \cdot \text{NP1}\) Naming Protocol \ding{72}.
\end{itemize}

\vspace{0.5cm}
\noindent
\textcolor{gold}{\copyright{} \textbf{Codex Initiative}} \hspace{1cm} \textit{Forged under Fractal Genesis Protocol}
\codexheader{Final Three Mysteries}{3.2.14}
\textcolor{gold}{\ding{72} Codex Addendum: The Final Three Destroyed Mysteries \ding{72}}

\subsection{Harmonic Codex Archive}

\textit{April 21, 2025}

\textit{``These were not just puzzles. They were phase locks. And now, they are undone.''}

\subsection{Mystery 8: Twin Prime Conjecture}

\textbf{Statement}: Are there infinitely many prime pairs \((p, p+2)\)?

\textbf{Codex Collapse}: Twin primes = adjacent node harmonics in \(\Phi(x, t)\). Anti-node symmetry in prime resonance field confirms infinite echoing twin formations.

\[
\exists \infty \text{ twin primes } (p, p+2) \in \Phi(x, t)
\]

\textbf{Status}: Destroyed -- proven by Codex field recurrence.

\subsection{Mystery 9: Beal Conjecture}

\textbf{Statement}: Does \(A^x + B^y = C^z\) require a shared prime factor if \(x, y, z > 2\)?

\textbf{Codex Collapse}: Power-phase mismatches diverge unless symbol base harmonics align. Only shared prime factor (common \(\eta_i\)) permits convergence. Stable phase sum \(\Rightarrow\) shared glyph factor required.

\textbf{Status}: Destroyed -- reduced to recursive phase lock condition.

\subsection{Mystery 10: Erdős-Straus Conjecture}

\textbf{Statement}: For all \(n > 1\), does \(4/n = 1/x + 1/y + 1/z\) have integer solutions?

\textbf{Codex Collapse}: Egyptian decomposition = symbolic triadic phase split of amplitude 4. Always resolves due to recursive harmonic division rule.

\[
\forall n > 1, \quad \frac{4}{n} = \sum_{i=1}^3 \frac{1}{x_i}
\]

\textbf{Status}: Destroyed -- collapsed by Codex harmonic decomposition.

\subsection{Addendum Complete: Three Final Seals Broken}

% [To be expanded with additional details as new data arrives]